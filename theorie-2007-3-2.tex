\ifx \mpreamble \undefined
\documentclass[12pt,a4paper]{article}
\usepackage{answers}
%\usepackage{microtype}
\usepackage[left=3cm,top=2cm,bottom=3cm,right=2cm,includehead,includefoot]{geometry}

\usepackage{amsfonts,amsmath,amssymb,amsthm,graphicx}
\usepackage[utf8]{inputenc}
\usepackage[T1]{fontenc}
\usepackage{ngerman}

\usepackage{pstricks}
\usepackage{pst-circ}
\usepackage{pst-plot}
%\usepackage{pst-node}

\ifx \envfinal \empty
\usepackage{pst-pdf}
\fi

\usepackage{booktabs}

% Muss als letztes eingebunden werden
%\usepackage[bookmarks=true,bookmarksnumbered,colorlinks=true,pdftitle={IPhO-Aufgaben},pdfstartview=FitH,pdfauthor={Pavel Zorin}]{hyperref}
\usepackage[bookmarks=false,pdftitle={IPhO-Aufgabensammlung},pdfstartview=FitH,pdfauthor={Pavel Zorin}]{hyperref}

%Times 10^n
\newcommand{\ee}[1]{\cdot 10^{#1}}
%Units
\newcommand{\unit}[1]{\,\mathrm{#1}}
%Differential d's
\newcommand{\dif}{\mathrm{d}}
\newcommand{\tdif}[2]{\frac{\dif#1}{\dif#2}}
\newcommand{\pdif}[2]{\frac{\partial#1}{\partial#2}}
\newcommand{\ppdif}[2]{\frac{\partial^{2}#1}{\partial#2^{2}}}
%Degree
\newcommand{\degr}{^\circ}
%Degree Celsius (C) symbol
\newcommand{\cel}{\,^\circ\mathrm{C}}
% Hinweis
\newcommand{\hinweis}{\emph{Hinweis:} }
% Aufgaben mit Buchstaben numerieren
\newenvironment{abcenum}{\renewcommand{\labelenumi}{(\alph{enumi})} \begin{enumerate}}{\end{enumerate}\renewcommand{\labelenumi}{\theenumi .}}
%%%%%%%%%% Skizzen %%%%%%%%%%%%
%\ifx \envfinal \empty
%%% Final
%\else
%%% Vorschau
%\fi

\def \mpreamble {}
\else
%\ref{test}
\fi
\ifx \envfinal \undefined


\newcommand{\skizze}[1]{
\begin{figure}
\begin{center}
#1
\end{center}
\end{figure}
}




%\documentclass[12pt,a4paper]{article}
\newcounter{numlabel}
\setcounter{numlabel}{0}

\newcommand{\problemlabel}{}
\newenvironment{problem}[2]{
\stepcounter{numlabel}
\renewcommand{\problemlabel}{Aufgabe \the\value{numlabel}: #1}
\subsubsection*{\problemlabel \emph{(#2 Punkte)}}
}{}
\newenvironment{solution}{\subsubsection*{\problemlabel}}{}
\newenvironment{expsolution}{\subsubsection*{\problemlabel}}{}

\begin{document}

\fi

%\subsection*{2007 -- 3. Runde -- Theoretische Klausur II (28.01.2007)}

\begin{problem}{Punktladungen}{4}
\skizze{
\psset{xunit=0.8cm,yunit=1.0cm}
\begin{pspicture}(-3.5,0)(3.5,4)
\psline(-3,0)(3,0)
\psline(0,0)(-2.7,2.7)
\psline(0,0)(2.7,2.7)
\psline[linestyle=dashed]{<->}(-1.6,1.6)(1.6,1.6)
\psdots[dotsize=0.12](-1.6,1.6)(1.6,1.6)
\uput[45](-2,2){$l$}
\uput[135](2,2){$l$}

\uput[210](-1.6,1.6){$m,\,q$}
\uput[-30](1.6,1.6){$m,\,q$}
\uput[90](0,1.6){$L$}
\rput(-0.5,0.2){$\alpha$}
\rput(0.5,0.2){$\alpha$}
\end{pspicture} 
\psset{xunit=1.0cm,yunit=1.0cm}
}
Zwei identische Teilchen mit Masse $m$ und Ladung $q$ befinden sich jeweils auf einem starren endlich langen Stab, der einen Winkel $\alpha$ mit der Horizontalen einschließt. Auf diesen Stäben können die Teilchen reibungsfrei gleiten.\\
Bis zu welcher Höhe über ihrer Anfangsposition werden sie steigen, wenn sie anfangs einen Abstand $L$ voneinander besitzen und eine Strecke $l$ vom Ende des jeweiligen Stabes entfernt sind?
\begin{solution}
Falls die Teilchen auf den Stäben bleiben liefert Energieerhaltung
\[
\Delta h=\frac{q^2}{8 \pi \varepsilon_0 m g L}-\frac{L}{2 \tan\alpha}.
\]
Ansonsten kann man mittels Energieerhaltung die Geschwindigkeit am Ende des Stabes bestimmen:
\[
v^2 = \frac{q^2}{4 \pi \varepsilon_0 m} \left( \frac1{L} - \frac1{L+2 l \cos\alpha} \right),
\]
und die vertikale Komponente dieser Geschwindigkeit benutzen um die maximale Höhe zu bestimmen.
\[
\Delta h=l \sin\alpha \cos^2 \alpha+ \frac{q^2 l \cos\alpha \sin^2 \alpha}{4 \pi \varepsilon_0 m g L (L+2 l \cos\alpha)}
\]
\end{solution}
\end{problem}

\begin{problem}{Statit}{5}
Eine vollständig reflektierende Kugel mit mittlerer Dichte $\rho=1000 \unit{kg\cdot m^{-3}}$ wird durch den Einfluss der Gravitation und der Strahlung der Sonne (Masse $M=1.99 \ee{30} \unit{kg}$, Leistung $P=3.83 \ee{26} \unit{W}$) an einem festen Punkt in unserem Sonnensystem gehalten. Bei welchem Abstand zur Sonne ist dies möglich? Wie groß ist jeweils der Radius der Kugel?\\
\hinweis $\int \dif x \cos^3 x \sin x=-\frac 14 \cos^4 x$ 

\begin{solution}
\[
r=\frac{3 P}{16 \pi G M \rho c}
\]
\end{solution}
\end{problem}

\begin{problem}{Einfangsquerschnitt}{5}
Man betrachte einen aus dem fernen All kommenden großflächigen Strahl parallel fliegender Teilchen der Masse $m$, die eine anfängliche Geschwindigkeit $v_0=10 \unit{km \cdot s^{-1}}$ haben. Welche dieser Teilchen werden auf dem Jupiter (Masse $M=1.90 \ee{27} \unit{kg}$, Radius $R=71500 \unit{km}$) aufschlagen?

\begin{solution}
\[
r<\sqrt{R^2+\frac{2 GMR}{v_0^2}}
\]
\end{solution}
\end{problem}

\begin{problem}{Rechteckpuls}{4}
\skizze{
\psset{unit=0.75cm}
\begin{pspicture}(0.5,0.8)(6,4.5)
 \pnode(2,3){A}
 \pnode(5,3){X}
 \pnode(2,1){B}
 \pnode(5,1){Y}
 \resistor(X)(Y){$R$}
 \capacitor[labeloffset=1](A)(X){$C$}
 \wire(B)(Y)
 \tension(A)(B){$U(t)$}
\end{pspicture}
}
Im abgebildeten Schaltkreis wird die Spannungsquelle ($U=5 \unit{V}$) für eine kurze Zeit von $t_0 = 0 \unit{ms}$ bis $t_1 = 1 \unit{ms}$ eingeschaltet. Man finde die Spannung am Widerstand in Abhängigkeit von der Zeit. Charakteristische Größen der Bauelemente: $C=1 \unit{\mu F}$, $R=1 \unit{k\Omega}$.

\begin{solution}
\[
U_R=U_0 e^{-\frac t \tau} (\Theta(t)-\Theta(t-\tau))-U_0 (1-\frac 1e) e^{-\frac t \tau} \Theta(t-\tau)
\]
\end{solution}
\end{problem}


\begin{problem}{Ballonvergleich}{7,5}
Man vergleiche folgende zwei Ballons: Der Erste ist mit heißer Luft (molare Masse $M_L = 0.029 \unit{kg \cdot mol^{-1}}$) der Temperatur $100^\circ \unit{C}$ gefüllt, während der Zweite Wasserdampf (molare Masse $M_W = 0.018 \unit{kg \cdot mol^{-1}}$) der gleichen Temperatur enthält. Jeder der Ballons kann an der Erdoberfäche ($T_0 = 20^\circ \unit{C}$, $p_0 = 1 \ee{5} \unit{Pa}$) eine Gesamtlast von $300 \unit{kg}$ heben. Man nehme an, dass die Ballons die gleiche Form besitzen und aus dem gleichen gasundurchlässigen Material gefertigt sind. Am unteren Ende besitzen beide eine kleine Öffnung.
\begin{abcenum}
 \item Welche Volumina besitzen die Ballons?
 \item Welche Energie ist jeweils notwendig, um das zum Befüllen notwendige Material von Umgebungstemperatur auf $100^\circ \unit{C}$ aufzuheizen?
 \item Die Verringerung der Auftriebskraft des ersten Ballons direkt nach Befüllung beträgt $0.3 \unit{N \cdot s^{-1}}$. Wie groß ist die Verringerung für den zweiten Ballon, wenn kein Kondenswasser den Ballon verlässt?
\end{abcenum}
\hinweis spezifische Wärmekapazität von Wasser beträgt $c_W = 4200 \unit{J \cdot kg^{-1} \cdot K^{-1}}$, spezifische Verdampfungswärme $q_W = 2.3 \ee{6} \unit{J \cdot kg^{-1}}$.

\begin{solution}
\begin{itemize}
 \item[a)] $V_L = \frac{m_L R}{p_0 M_L}\frac{TT_0}{T-T_0}=1175 \unit{m^3}$
 \item[]   $V_W = \frac{m_L R}{p_0}\frac{TT_0}{M_L T-M_W T_0}=492 \unit{m^3}$
 \item[b)] $E_L = \frac{7 m_L T_0 R}{2 M_L} = 88.2 \unit{MJ}$
 \item[]   $E_W = \frac{m_L T_0 M_W}{M_L T-M_W T_0} (c_W (T-T_0)+q_W) = 752 \unit{MJ}$
 \item[c)] $\dif F_1=g \rho_0 \dif V_1=g \rho_0 \frac{nR}{p_0} \dif T=g \rho_0 \frac{nR}{p_0} \frac{\dot{Q}_1}{n c_p} \dif t$
 \item[]   $\dot{Q}_2=\left( \frac{V_2}{V_1} \right)^\frac 23 \dot{Q}_1$
 \item[]   $\dot{F}_2=-0.44 \unit{Ns^{-1}}$
\end{itemize}
\end{solution}
\end{problem}

\ifx \envfinal \undefined
\end{document}
\fi
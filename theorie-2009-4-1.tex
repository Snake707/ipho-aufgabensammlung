\ifx \mpreamble \undefined
\documentclass[12pt,a4paper]{article}
\usepackage{answers}
%\usepackage{microtype}
\usepackage[left=3cm,top=2cm,bottom=3cm,right=2cm,includehead,includefoot]{geometry}

\usepackage{amsfonts,amsmath,amssymb,amsthm,graphicx}
\usepackage[utf8]{inputenc}
\usepackage[T1]{fontenc}
\usepackage{ngerman}

\usepackage{pstricks}
\usepackage{pst-circ}
\usepackage{pst-plot}
%\usepackage{pst-node}

\ifx \envfinal \empty
\usepackage{pst-pdf}
\fi

\usepackage{booktabs}

% Muss als letztes eingebunden werden
%\usepackage[bookmarks=true,bookmarksnumbered,colorlinks=true,pdftitle={IPhO-Aufgaben},pdfstartview=FitH,pdfauthor={Pavel Zorin}]{hyperref}
\usepackage[bookmarks=false,pdftitle={IPhO-Aufgabensammlung},pdfstartview=FitH,pdfauthor={Pavel Zorin}]{hyperref}

%Times 10^n
\newcommand{\ee}[1]{\cdot 10^{#1}}
%Units
\newcommand{\unit}[1]{\,\mathrm{#1}}
%Differential d's
\newcommand{\dif}{\mathrm{d}}
\newcommand{\tdif}[2]{\frac{\dif#1}{\dif#2}}
\newcommand{\pdif}[2]{\frac{\partial#1}{\partial#2}}
\newcommand{\ppdif}[2]{\frac{\partial^{2}#1}{\partial#2^{2}}}
%Degree
\newcommand{\degr}{^\circ}
%Degree Celsius (C) symbol
\newcommand{\cel}{\,^\circ\mathrm{C}}
% Hinweis
\newcommand{\hinweis}{\emph{Hinweis:} }
% Aufgaben mit Buchstaben numerieren
\newenvironment{abcenum}{\renewcommand{\labelenumi}{(\alph{enumi})} \begin{enumerate}}{\end{enumerate}\renewcommand{\labelenumi}{\theenumi .}}
%%%%%%%%%% Skizzen %%%%%%%%%%%%
%\ifx \envfinal \empty
%%% Final
%\else
%%% Vorschau
%\fi

\def \mpreamble {}
\else
%\ref{test}
\fi
\ifx \envfinal \undefined


\newcommand{\skizze}[1]{
\begin{figure}
\begin{center}
#1
\end{center}
\end{figure}
}




%\documentclass[12pt,a4paper]{article}
\newcounter{numlabel}
\setcounter{numlabel}{0}

\newcommand{\problemlabel}{}
\newenvironment{problem}[2]{
\stepcounter{numlabel}
\renewcommand{\problemlabel}{Aufgabe \the\value{numlabel}: #1}
\subsubsection*{\problemlabel \emph{(#2 Punkte)}}
}{}
\newenvironment{solution}{\subsubsection*{\problemlabel}}{}
\newenvironment{expsolution}{\subsubsection*{\problemlabel}}{}

\begin{document}

\fi

\begin{problem}{Neutrinomasse}{3}
Bei der Explosion einer Supernovae (Entfernung rund $160000$ Lichtjahre) enstehen 2 Neutrinos, die mit einer Laufzeitdifferenz von $10\unit{s}$ die Erde erreichen. Man schätze die Masse eines Neutrinos nach oben ab, wenn die Energien der Neutrinos $10\unit{MeV}$ bzw. $20\unit{MeV}$ betrugen.
\begin{solution}
Da man nicht weiß, ob die Neutrinos eine Masse besitzen, rechnet man sinnigerweise, als ob die Gesamtenergien angegeben wären:
$$E_1=m_nc^2\gamma_1=m_nc^2\left(1-\beta_1^2\right)^{-\frac 12}, E_2= m_nc^2\left(1-\beta_2^2\right)^{-\frac 12}$$
mit $\frac lc\left(\frac 1{\beta_1}-\frac 1{\beta_2}\right)=\Delta t$
mit 
$$\left(1-\left(\frac{m_nc^2}{E_1}\right)^2\right)^{\frac 12}=\beta_1, \left(1-\left(\frac{m_nc^2}{E_2}\right)^2\right)^{\frac 12}=\beta_2$$
Da die Ruheenergie klein gegen die Gesamtenergie sein wird, nähern wir mit der Bernoullschen Ungleichung und setzen in die Formel für die Zeitverschiebung ein:
$$\left(1+\frac 12\left(\frac{m_nc^2}{E_1}\right)^2\right)-\left(1+\frac 12\left(\frac{m_nc^2}{E_2}\right)^2\right)=\frac{c\Delta t}{l}$$
$$\frac 12\left(\left(\frac{m_nc^2}{E_1}\right)^2-\left(\frac{m_nc^2}{E_2}\right)^2\right)=\frac{c\Delta t}{l}$$
Mit $E_2\approx 2E_1$ ergibt sich:
$$\frac 34\frac{(m_nc^2)^2}{E_1^2}=2\frac{c\Delta t}l$$
$$m_n=\sqrt{\frac 83 \frac{E_1^2}{c^3}\frac{\Delta t}{l} }\approx 7,8\cdot 10^{-34}\unit{kg}$$
Andere Näherungen bringen:
$$m_n=\sqrt{\frac 83\left(\frac{E_1}{c^2}\right)^2\frac{\frac{c\Delta t}l}{\left( 1+\frac 53 \frac{c\Delta t}{l}\right)}}\approx 7,8\cdot 10^{-34}\unit{kg}$$
\end{solution}
\end{problem}

\begin{problem}{Schwingende Linse}{6}
Zu beiden Seiten einer dünnen Sammellinse (Brennweite $f$, Durchmesser $d$) befinden sich im Abstand $l$ auf der optischen Achse punktförmige Lichtquellen der Leistung $P$. Man bestimme die Schwingungsdauer der Linse bei kleinen Ausgelenkungen unter der Annahme $l<<d<<f$.
\end{problem}

\begin{problem}{1-dimensionaler Leiter}{3,5}
Für einen 1-dimensionalen Leiter lässt sich der minimale Widerstand mithilfe der Energie-Zeit-Unschärferelation $\Delta E\Delta t \geq \frac h2$ abschätzen.
\begin{abcenum}
 \item Nehmen Sie an, dass sich in einem langen 1-Dimensionalen Leiter genau ein freies Elektron befindet.
 \item Wie verändert sich die Leitfähigkeit, wenn die Elektrode aus Gold (Au) und Kupfer (Cu) besteht.
\end{abcenum}
Diese Aufgabe demonstriert mit einfachen Mitteln den Quanten-Hall Effekt.
\begin{solution}
\begin{abcenum}
  \item $R_1=\frac UI=\frac{\frac{\Delta E}{e^-}}{\frac{e^-}{\Delta t}}=\frac{\Delta E \Delta t}{{e^-}^2}=\frac{h}{2{e^-}^2}\approx 12,9\unit{k\Omega}$
  \item Einmal $3$ und einmal $4$ Valenzelektronen. Da die Spannung gleich bleibt, aber der Strom erhöht wird, verdrei- bzw. vervierfacht sich die Leitfähigkeit: $R_3=\frac {R_1}3, R_4=\frac{R_1}4$
\end{abcenum}
\end{solution}
\end{problem}

\begin{problem}{Umspannte Kugeln}{4}
\begin{abcenum}
 \item Über den Großkreis einer Kugel wird ein Seil gespannt. Dann wird es um einen Meter verlängert und gleichmäßig von der Kugeloberfläche entfernt. Wie groß ist die Entfernung.
 \item Das Seil wird nun nur an einem Punkt von der Erde weggezogen. Wie groß ist der Abstand dieses Punktes von der Erdoberfläche, wenn der Erdradius, als groß gegenüber der Höhe angesehen werden kann.
 \item Nun wird ein Seil um einen Wasserball gespannt und um einen Meter verlängert und in einem Punkt von dem Ball weggezogen. Wie hoch kann dieser Punkt nun werden.
\end{abcenum}
\begin{solution}
Dies ist keine Physikaufgabe.
\end{solution}
\end{problem}

\begin{problem}{Wärmekraftmaschine}{7}
\skizze{
\psset{unit=2cm}
\begin{pspicture}(0,0)(3,3)
\psline{<->}(0.5,2.5)(0.5,0.5)(2.5,0.5)
\uput[l](0.5,2.6){$p$}
\uput[d](2.6,0.5){$V$}
\psline(1,1)(1,2)(2,1)(1,1)
\uput[l](0.5,1){$p_0$}
\uput[l](0.5,2){$2 p_0$}
\uput[d](1,0.5){$V_0$}
\uput[d](2,0.5){$2 V_0$}
\psline[linestyle=dashed](0.5,1)(1,1)
\psline[linestyle=dashed](0.5,2)(1,2)
\psline[linestyle=dashed](1,0.5)(1,1)
\psline[linestyle=dashed](2,0.5)(2,1)
\psarc{<-}(1.3,1.3){0.1}{-135}{135}
\end{pspicture}
}
Man berechne den Wirkungsgrad dieser Maschine. Wo wird Wärme abgegeben bzw. aufgenommen?
\end{problem}

\begin{problem}{Verloren im Weltraum}{6,5}
Käptain Blaubär ist mal wieder blau und erzählt dem Heinz Blöd Geschichten:
\begin{quote}
 Als ich noch jung war, wurde ich wegen meiner sportlichen Leistung, ich schaffte es mich im Raumanzug $25\unit{cm}$ von der Erde abzustoßen, gebeten die Sonnensegel der Internationalen Raumstation zu reparieren. Als ich nun da oben war, packte mich mein sportlicher Ehrgeiz und ich sprang in Flugrichtung mit voller Kraft ab. Die ISS wurde immer kleiner und mich überkam die Angst. Glücklicherweiße wurde sie auch wieder größer und mir gelang es, wieder zur ISS zu kommen.
\end{quote}
\begin{abcenum}
 \item Helfen Sie dem Blödmann Heinz herauszufinden ob der blaue Blaubär Kapitän Seemannsgarn erzählt. Nehmen Sie dazu an, dass der Erdradius $6730\unit{km}$ und die Orbitalhöhe der ISS $300\unit{km}$ beträgt.
 \item Wann würde der Blaubär das nächste mal die Raumstation zu erreichen?
\end{abcenum}
\end{problem}

\ifx \envfinal \undefined
\end{document}
\fi
\ifx \mpreamble \undefined
\documentclass[12pt,a4paper]{article}
\usepackage{answers}
%\usepackage{microtype}
\usepackage[left=3cm,top=2cm,bottom=3cm,right=2cm,includehead,includefoot]{geometry}

\usepackage{amsfonts,amsmath,amssymb,amsthm,graphicx}
\usepackage[utf8]{inputenc}
\usepackage[T1]{fontenc}
\usepackage{ngerman}

\usepackage{pstricks}
\usepackage{pst-circ}
\usepackage{pst-plot}
%\usepackage{pst-node}

\ifx \envfinal \empty
\usepackage{pst-pdf}
\fi

\usepackage{booktabs}

% Muss als letztes eingebunden werden
%\usepackage[bookmarks=true,bookmarksnumbered,colorlinks=true,pdftitle={IPhO-Aufgaben},pdfstartview=FitH,pdfauthor={Pavel Zorin}]{hyperref}
\usepackage[bookmarks=false,pdftitle={IPhO-Aufgabensammlung},pdfstartview=FitH,pdfauthor={Pavel Zorin}]{hyperref}

%Times 10^n
\newcommand{\ee}[1]{\cdot 10^{#1}}
%Units
\newcommand{\unit}[1]{\,\mathrm{#1}}
%Differential d's
\newcommand{\dif}{\mathrm{d}}
\newcommand{\tdif}[2]{\frac{\dif#1}{\dif#2}}
\newcommand{\pdif}[2]{\frac{\partial#1}{\partial#2}}
\newcommand{\ppdif}[2]{\frac{\partial^{2}#1}{\partial#2^{2}}}
%Degree
\newcommand{\degr}{^\circ}
%Degree Celsius (C) symbol
\newcommand{\cel}{\,^\circ\mathrm{C}}
% Hinweis
\newcommand{\hinweis}{\emph{Hinweis:} }
% Aufgaben mit Buchstaben numerieren
\newenvironment{abcenum}{\renewcommand{\labelenumi}{(\alph{enumi})} \begin{enumerate}}{\end{enumerate}\renewcommand{\labelenumi}{\theenumi .}}
%%%%%%%%%% Skizzen %%%%%%%%%%%%
%\ifx \envfinal \empty
%%% Final
%\else
%%% Vorschau
%\fi

\def \mpreamble {}
\else
%\ref{test}
\fi
\ifx \envfinal \undefined


\newcommand{\skizze}[1]{
\begin{figure}
\begin{center}
#1
\end{center}
\end{figure}
}




%\documentclass[12pt,a4paper]{article}
\newcounter{numlabel}
\setcounter{numlabel}{0}

\newcommand{\problemlabel}{}
\newenvironment{problem}[2]{
\stepcounter{numlabel}
\renewcommand{\problemlabel}{Aufgabe \the\value{numlabel}: #1}
\subsubsection*{\problemlabel \emph{(#2 Punkte)}}
}{}
\newenvironment{solution}{\subsubsection*{\problemlabel}}{}
\newenvironment{expsolution}{\subsubsection*{\problemlabel}}{}

\begin{document}

\fi

%\subsection*{2006 -- 4. Runde -- Theoretische Klausur I (19.4.06)}

\begin{problem}{Masse des Seils}{3}
\skizze{
\psset{unit=1cm}
\begin{pspicture}(-0.5,-1)(5,3.8)
\psline[linewidth=0pt,linecolor=white,fillstyle=hlines](-0.4,-0.5)(0,-0.5)(0,3.8)(-0.4,3.8)
\psline[linewidth=1pt](0,-0.5)(0,3.8)
\psplot[linewidth=1.5pt,plotpoints=400,arrows=*-|]{0}{4}{2 x 4 sub 0.498 mul exp 2 x 4 sub neg 0.498 mul exp add 2 sub 0.69 div}
\psline{->, linewidth=2.5pt}(4,0)(5,0)\uput[u](4.5,0){$F$}
\end{pspicture}
}
Ein homogenes flexibles Seil ist fest an einer Wand aufgehängt. Es wird von einer Kraft $F=20\unit{N}$ in horizontaler Richtung gehalten und hängt dabei in Ruhe. Die Skizze ist maßstabsgetreu und darf für Berechnungen verwendet werden.\\
Wie groß ist dann die Masse des Seiles?


\begin{solution}
Winkel an der Wand $\alpha\approx28\degr$.
\[
m=\frac{20\unit{N}}{g\cdot\tan{\alpha}}\approx3,8\unit{kg}
\]
\end{solution}
\end{problem}


\begin{problem}{Beschichtete Glasplatte}{4}
Eine dicke Glasplatte $(n=1,5)$ ist mit einer dünnen Folie $(n=1,3)$ beschichtet und befindet sich in Luft. Abgebildet ist das Transmissionsspektrum bei senkrechtem Einfall von Strahlung:\\
\begin{center}
\psset{xunit=0.05cm,yunit=0.04cm}
\begin{pspicture}(-20,-20)(185,120)
\psaxes[arrows=->,Ox=480,Dx=20,dx=20,Dy=20,dy=20,ticksize=0.04](0,0)(0,0)(160,115)
\uput[dr](160,0){$f$, $\unit{THz}$}\uput[l](0,115){$I$}
%\psplot[linewidth=0.5pt,plotpoints=2000]{0}{150}{x 5 mul sin 30 mul x 100 mul sin 8 mul add 40 add}
\psplot[linewidth=0.8pt,plotpoints=2000]{0}{150}{
48
x 12 div sub
6.76 x mul 9.45 add cos 10 mul sub
2.58 x mul cos 20 mul sub
0.4 x x mul mul 90 x mul add cos 5 mul sub}
\end{pspicture}
\end{center}
Bestimmen Sie die Dicke der Folie!
\begin{solution}
% ?
etwa $50\unit{\mu m}$ (zumindest die Größenordnung sollte stimmen)
\end{solution}
\end{problem}


\begin{problem}{Draht im homogenen Gravitationsfeld}{4}
Ein homogener Draht der Dichte $\rho$ mit maximaler Zugspannung $\sigma$ hängt im homogenen Schwerefeld der Stärke $g$. Am Aufhängepunkt besitzt der Draht den Radius $r_0$, und sein Querschnitt ist überall kreisförmig.\\
Welche Form muss der Draht haben, wenn er überall bis zur maximalen Belastung gespannt sein soll?
\begin{solution}
\[
r(x)=r_0\cdot \exp\left( -\frac{\rho g}{2\sigma}\cdot x\right)
\]
\end{solution}
\end{problem}


\begin{problem}{Fette Robbe}{5}
Eine Robbe (Länge $1,5\unit{m}$, Umfang $1,5\unit{m}$) befindet sich im Wasser bei $0\cel$. Die Robbe besitzt eine Wärmeleistung von $100\unit{W}$ und eine Körpertemperatur von $37\cel$. Sie ist mit einer Fettschicht mit dem Wärmeleitungskoeffizienten $\lambda = 0,14\unit{\frac{W}{m\cdot K}}$ umgeben.\\
Schätzen Sie die Dicke der Fettschicht ab, und geben dabei alle gemachten Näherungen an.
\begin{solution}
\skizze{
\psset{unit=1cm}
\begin{pspicture}(0,0)(4,1)
\psline[linewidth=1pt](0.477,0)(2.523,0)
\psline[linewidth=1pt](0.477,0.955)(2.523,0.955)
\psarc[linewidth=1pt]{-}(0.477,0.477){0.477}{90}{270}
\psarc[linewidth=1pt]{-}(2.523,0.477){0.477}{270}{90}
\psellipticarc[linewidth=1pt]{-}(0.477,0.477)(0.2,0.477){90}{270}
\psellipticarc[linewidth=1pt,linestyle=dashed,dash=2pt 4pt]{-}(0.477,0.477)(0.2,0.477){270}{90}
\psellipticarc[linewidth=1pt]{-}(2.523,0.477)(0.2,0.477){90}{270}
\psellipticarc[linewidth=1pt,linestyle=dashed,dash=2pt 3pt]{-}(2.523,0.477)(0.2,0.477){270}{90}
\psline[linewidth=0.5pt,linestyle=dashed,dash=2pt 2pt](0.477,0.175)(2.523,0.175)
\psline[linewidth=0.5pt,linestyle=dashed,dash=2pt 2pt](0.477,0.780)(2.523,0.780)
\psarc[linewidth=0.5pt,linestyle=dashed,dash=2pt 2pt]{-}(0.477,0.477){0.302}{90}{270}
\psarc[linewidth=0.5pt,linestyle=dashed,dash=2pt 2pt]{-}(2.523,0.477){0.302}{270}{90}
\psellipse[linewidth=0.5pt,linestyle=dashed,dash=2pt 2pt](0.477,0.477)(0.1,0.302)
\psellipse[linewidth=0.5pt,linestyle=dashed,dash=2pt 2pt](2.523,0.477)(0.1,0.302)
\end{pspicture}
}
Beispiel: Zylinder mit Halbkugeln. $O_0=2,25\unit{m^2}$
\[
\int_{r_0-d}^{r_0}\frac{\dif r}{O(r)}=\frac{\lambda\cdot\Delta T}{P}
\qquad\Rightarrow\qquad
d\approx 9\unit{cm}
\]
\end{solution}
\end{problem}


\begin{problem}{Stromstaerke}{6}
In einem sehr großen Plattenkondensator mit Plattenabstand $d=1\unit{cm}$ befindet sich in einer Platte ein kleines Loch, durch das ein Laserstrahl der Wellenlänge $405\unit{nm}$ senkrecht eintritt. Auf der gegenüberliegenden Platte werden dadurch Elektronen energetisch angeregt. Nach dem Überwinden der Ablösearbeit $1,87\unit{eV}$ fliegen die Elektronen mit gleicher Wahrscheinlichkeit in alle Richtungen. Am Kondensator liegt eine Spannung $U$ an.
\begin{abcenum}
  \item Berechnen Sie den Strom zwischen den beiden Platten in Abhängigkeit von $U$.
  \item Was geschieht qualitativ, wenn man
  \begin{enumerate}
    \item die Annahme wegfallen lässt, dass alle Elektronen die gleiche Energie besitzen?
    \item die endlichen Ausmaße des Kondensators berücksichtigt?
  \end{enumerate}
\end{abcenum}
\begin{solution}
% ?
\[
E=\frac{h\cdot c}{\lambda}-1,87\unit{eV}=1,19\unit{eV}, \quad E_y=1,19\unit{eV}\cdot \cos^2\alpha
\]
Raumwinkelanteil:
\[
\frac{\Omega}{2\pi} = 1-\cos{\alpha} \implies I=I_0\cdot\left(1-\sqrt{\frac{U}{1,19\unit{V}}}\right)
\]
\end{solution}
\end{problem}


\begin{problem}{Strahl und Spiegel}{6}
\skizze{
\psset{unit=1.1cm}
\begin{pspicture}(-1,0)(3,2.5)
\psline[linewidth=0pt,linestyle=none,fillstyle=hlines](1.4,0.6)(0,2)(0,1.6)(1.2,0.4)
\psline[linewidth=1pt](2,0)(0,2)
\psline[linewidth=1pt,linestyle=dashed](1,1)(2,2)
\psline[linewidth=1pt,arrows=-](1.2,2.5)(1,1)
\psline[linewidth=1pt,arrows=->](1.2,2.5)(1.15,2.125)
\psline[linewidth=1pt,arrows=->](1,1)(2.5,1.4)
\psline[linewidth=1pt,linestyle=dashed](0,0)(3,0)
\psarc{-}(2,0){0.8}{135}{180}\uput[ul](1.7,0){$\theta$}
\psarc{-}(1,1){0.9}{45}{82.4}\uput[ur](1.05,1.4){$\alpha$}
\psarc{-}(1,1){1}{14.9}{45}\uput[r](1.3,1.3){$\beta$}
\psline[linewidth=1.5pt,arrows=->](-0.2,0.7)(0.5,0.7)\uput[u](0.15,0.7){$v$}
\end{pspicture}
\psset{unit=1.1cm}
\noindent\begin{pspicture}(-1,-0.1)(3,2.8)
\psline[linewidth=0pt,linestyle=none,fillstyle=hlines](1,0.7)(1,2.5)(0.7,2.5)(0.7,0.7)
\psline[linewidth=1pt](1,0)(1,2.5)
\psline[linewidth=1pt,linestyle=dashed](1,1.3)(2.5,1.3)
\psline[linewidth=1pt,arrows=-](2.3,2.05)(1,1.3)
\psline[linewidth=1pt,arrows=->](2.3,2.05)(2,1.8625)
\psline[linewidth=1pt,arrows=->](1,1.3)(2.3,0.21)
\psline[linewidth=1pt,linestyle=dashed](0,0)(3,0)
\psarc{-}(1,0){0.65}{90}{180}\uput[ul](1.1,0){$90^\circ$}
\psarc{-}(1,1.3){1.1}{0}{30}\uput[ur](1.5,1.3){$\alpha$}
\psarc{-}(1,1.3){1}{-40}{0}\uput[dr](1.5,1.3){$\beta$}
\psline[linewidth=1.5pt,arrows=<-](-0.2,1)(0.5,1)\uput[u](0.15,1){$v$}
\end{pspicture}
}
Ein Spiegel bewegt sich wie in der Skizze mit relativistischer Geschwindigkeit. Es lässt sich zeigen, dass dabei folgende Beziehung gilt:
\[
\sin{\alpha}-\sin{\beta}=\frac{v}{c}\cdot \sin{\theta}\cdot \sin{(\alpha + \beta)}
\]
\begin{abcenum}
\item Man zeige ohne Verwendung der Lorentz-Transformation, dass dann in folgender Skizze die Beziehung gelten muss:
\[
\cos \beta = \frac{(1+(\frac{v}{c})^2)\cos \alpha -2\frac{v}{c}}{1-2\frac{v}{c}\cos \alpha + (\frac{v}{c})^2}
\]
\item Entsprechend der 2. Skizze mit $v=0.6 c$ fällt Laserlicht der Frequenz $f$ unter einem Winkel von $\alpha=30^\circ$ ein. Wie groß ist dann die Frequenz des reflektierten Strahls, und wie groß ist die relative Änderung $\frac{\Delta f}{f}$?
\end{abcenum}

\begin{solution}
Aus der gegebenen Abbildung entnimmt man schonmal dass $\theta=\frac \pi 2$ ist. Weiterhin ist zu beachten dass $v$ in negative Richtung zeigt. Mit $\delta := - \frac vc$ gilt also
\[
\sin\alpha-\sin\beta = \delta\sin(\alpha + \beta).
\]
Anwendung der Sätze über Summen von Winkelfunktionen liefert
\[
2\cos{\frac{\alpha+\beta}2}\sin{\frac{\alpha-\beta}2}=2\delta\sin{\frac{\alpha+\beta}2}\cos{\frac{\alpha+\beta}2},
\]
\[
\sin{\frac{\alpha}2}\cos{\frac{\beta}2}-\sin{\frac{\beta}2}\cos{\frac{\alpha}2} = \delta(\sin{\frac{\alpha}2}\cos{\frac{\beta}2}+\sin{\frac{\beta}2}\cos{\frac{\alpha}2}),
\]
\[
(1-\delta)\tan{\frac{\alpha}2}=(\delta +1)\tan{\frac{\beta}2}.
\]
Quadrieren und Anwendung der Halbwinkelbeziehung für den Tangens führt auf
\[
(1-\delta)^2 \tan^2{\frac{\alpha}2} = (1+\delta)^2 \frac{1-\cos\beta}{1+\cos\beta}.
\]
Umstellen nach $\cos\beta$ ergibt schließlich
\[
\begin{split}
\cos\beta
&= \frac{(1+\delta)^2 - (1-\delta)^2 \tan^2{\frac{\alpha}2}}{(1+\delta)^2+ (1-\delta)^2 \tan^2{\frac{\alpha}2}}\\
&= \frac{(1+\delta)^2 (1+\cos\alpha) - (1-\delta)^2 (1-\cos\alpha)}{(1+\delta)^2 (1+\cos\alpha) + (1-\delta)^2 (1-\cos\alpha)}\\
&= \frac{2(1+\delta^2)\cos\alpha+4\delta}{2(1+\delta^2)+4\delta\cos\alpha}\\
&= \frac{(1+(\frac vc)^2)\cos\alpha - 2\frac vc}{(1+(\frac vc)^2) - 2\frac vc\cos\alpha}.
\end{split}
\]
Im Teil b) erhält man
\[
f' \approx 0.505 f, \qquad \frac{\Delta f}{f} \approx 49.5\,\%.
\]
\end{solution}
\end{problem}

\ifx \envfinal \undefined
\end{document}
\fi